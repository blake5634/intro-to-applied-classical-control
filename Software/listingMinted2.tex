\documentclass{article}

% Required packages
\usepackage[T1]{fontenc}
\usepackage{xcolor}
\usepackage{minted}
\usepackage{caption}


\definecolor{bg}{RGB}{220,220,220}

% Configure minted styling
\setminted[python]{
    frame=lines,
%     bgcolor=bg,
    framesep=0mm,
    linenos=true,
    baselinestretch=1.0,
    fontsize=\footnotesize,
    breaklines=true
}

% Configure listing caption style
\DeclareCaptionFormat{listing}{\raggedright#1#2#3}
\captionsetup[listing]{
    format=listing,
    labelfont=bf,
    font=small,
    labelsep=period
}

\begin{document}

\section{Python Code Examples}

Here is our first code listing:

\begin{listing}
\begin{minted}{python}
def hello():
    """A simple greeting function"""
    print("Hello world")
    return True
\end{minted}
\caption{Hello World function implementation}
\label{lst:hello}
\end{listing}

As we can see in Listing~\ref{lst:hello}, the function returns True after printing.
Let's look at another example:

\begin{listing}
\begin{minted}{python}
def fibonacci(n):
    """Calculate the nth Fibonacci number recursively"""
    if n <= 1:
        return n
    return fibonacci(n-1) + fibonacci(n-2)

# Example usage
result = fibonacci(10)
print(f"The 10th Fibonacci number is: {result}")
\end{minted}
\caption{Recursive Fibonacci implementation}
\label{lst:fibonacci}
\end{listing}

The recursive implementation in Listing~\ref{lst:fibonacci} shows a classic
approach to calculating Fibonacci numbers, though it's not the most efficient
for large values of n.

\end{document}
