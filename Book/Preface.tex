\chapter*{Preface}

Initial Github distribution: August 2019. 
\vspace{0.25in}

This book evolved from a set of course notes from teaching EE447,
a 10-week course on mostly classical control
in the Department of Electrical and Computer Engineering at the University of Washington.
My aims in the course was to enable non-specialist EE (now ECE) graduates to 
conver the minimal set of concepts needed to apply 
classical control design methods to basic control problems and implement a computer
based controller. 

The strategy is thorough practice and coverage of a minimal set of concepts rather than 
racing through a comprehensive syllabus. 

Computing is based around the open source Scilab package.  Many programs are based heavily 
on Mathwork's excellent Matlab software but are considering open source alternatives.   
Scilab has the advantage of being close in style and syntax to Matlab.  Starting 
over from today, I would reframe this with python, but if students are starting from scratch
in python, and have Matlab exposure from previous courses, there is not really time 
in a 10 week course.   At least a week would have to be devoted.   

Because of the small dimensionality of the controllers designed in this course, ``brute-force" 
optimization is employed.  The power of today's computers make this attractive and avoids 
extensive attention to starting point and local optimum worries. 

Problems are provided for active learning.   In teaching the course, I spent about 1/2 the contact
hours supporting students while they worked on the in-class-problems (``ICPs").   Then traditional 
homework problems done individually solidified the knowledge gained. 

