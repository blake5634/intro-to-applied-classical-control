
%%%%%%%%%%%%%%%%%%%%%%%%%%%%%%%%%%%%%%%%%%%%%%%%%%%%%%
%
%  Example environment
%
\newcounter{Example}[chapter]

\newenvironment{Example}
% begin
{\refstepcounter{Example}\newpage
 \begin{boxedminipage}{\textwidth} % \linenumbers
    {\bf Example \thechapter.\theExample}\\
}
% end
{\vspace{0.1in}\end{boxedminipage}
\vspace{0.4in} }

\newenvironment{ExampleCont}
% begin
{
 \begin{boxedminipage}{\textwidth}\linenumbers
    {\bf Example \thechapter.\theExample \hspace{4pt} cont.}\\
}
% end
{\vspace{0.1in}\end{boxedminipage}
\vspace{0.4in} }


\newenvironment{ExampleSmall}
% begin
{\refstepcounter{Example}
\vspace{0.2in}
 \begin{boxedminipage}{\textwidth}
    {\bf Example \thechapter.\theExample}\\
}
% end
{\vspace{0.05in}\end{boxedminipage}
\vspace{0.25in} }
%
%%%%%%%%%%%%%%%%%%%%%%%%%%%%%%%%%%%%%%%%%%%%%%%%%%%%%


%%%%%%%%%%%%%%%%%%%%%%%%%%%%%%%%%%%%%%%%%%%%%%%%%%%
%
%    Minted package for nice python highlighting
%
\usepackage[chapter]{minted}  % python syntax highlight

% Required packages

% Configure minted styling
\setminted[python]{
    frame=lines,
    framesep=2mm,
    baselinestretch=1.0,
    fontsize=\footnotesize,
    linenos=true,
    breaklines,
    %     style=monokai,
    style=default
}
\setminted[C]{
    frame=lines,
    framesep=2mm,
    baselinestretch=1.0,
    fontsize=\footnotesize,
    linenos=true,
    breaklines,
    %     style=monokai,
    style=default
}

% Configure listing to have a specific width
\setminted{
    numbersep=2pt,    % Distance between line numbers and code
    xleftmargin=20pt, % Left margin (increase if line numbers still overflow)
    xrightmargin=-2pt  % Right margin (decrease to reduce extra white space)
}

% Configure listing caption style
\DeclareCaptionFormat{listing}{\raggedright#1#2#3}
\captionsetup[listing]{
    format=listing,
    labelfont=bf,
    font=small,
    labelsep=period
}

%%%%%%% end of minted %%%%%%%%%%%%%%%%%%%%%%%%%%%%%%%%%%%%%%%%%%%%


%%%%%%%%%%%%%%%%%%%%%%%%%%%%%%%%%%%%%%%%%%%%%%%%%%%%%%%%%%%%%%
%       Claude formatting code begins here
\definecolor{claudeBlue}{RGB}{59, 89, 152}
\definecolor{claudeGray}{RGB}{242, 244, 248}

\newcommand{\humanquery}[1]{%
  \par\vspace{\baselineskip}%
  \noindent%
  \fbox{%
    \parbox{\dimexpr\linewidth-2\fboxsep-2\fboxrule\relax}{#1}%
  }%
  \par\vspace{\baselineskip}%
}

\newcommand{\claudereply}[1]{%
  \par\vspace{\baselineskip}%
  \noindent%
  \fcolorbox{claudeBlue}{claudeGray}{%
    \parbox{\dimexpr\linewidth-2\fboxsep-2\fboxrule\relax}{%
      \ttfamily\raggedright #1%
    }%
  }%
  \par\vspace{\baselineskip}%
}
%%%%%%%%%%%%%%%%%%%%%%%%%%%%%%% Claude formatting code ends here


%
%   quick type equation envir.
%
\def\bq{\begin{equation}}
\def\eq{\end{equation}}

% Symbol defs for equations
\def\ef{\mathcal{E}}
\def\fl{\mathcal{F}}

% Laplace Transform
\newcommand{\sL}{\mathcal{L}}


